\graphicspath{{introduction/fig/}}

\chapter{Introduction}
\label{chap:introduction}

During the COVID-19 lockdown in 2020, gardening became a popular method of relieving stress \cite{covid_stress_study} and improving mental health and resilience \cite{mental_resiliance_study}. Additionally, gardening provides physical and psychological benefits to older adults \cite{seniors_psych_study}, many of whom have to adjust their gardening practices as they age and become unable to safely perform some tasks \cite{seniors_garden_study}.

Automating crucial tasks would allow individuals to start and maintain healthy, productive gardens despite the inability to reliably maintain them manually.
\\

This paper focuses specifically on indoor herb gardens as monitoring and maintaining larger outdoor gardens requires larger, more complex systems that would require significantly more time and resources to develop.
\\

Soil moisture levels have been shown to be an important factor in the distribution, growth and development of herbs, along with air humidity \cite{humidity_moisture_study}. Additionally, soil moisture levels influence the root/shoot ratio of herbs. Higher levels of soil moisture lead to a larger percentage of herb growth occurring above ground \cite{root_shoot_study}. In the case of herbs grown for cooking, this is ideal as it leads to more usable leaves. 
\\

A device that is able to automatically monitor and maintain optimal ground moisture levels will therefore provide the largest benefit to maintaining a healthy indoor herb garden.

\section{Project Objectives}
This project aims to design, build and test a device that will:
\begin{itemize}
    \item Monitor ground moisture levels
    \item Monitor exposure to sunlight
    \item Automatically water the soil to maintain optimal moisture levels
    \item Display recorded data and allow the user to adjust desired soil moisture levels using a mobile application
    \item Reliably function despite power outages
\end{itemize}

